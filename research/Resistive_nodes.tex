\documentclass{article}

\usepackage{amsmath}
\usepackage{graphicx}
\usepackage{caption,subcaption}
\usepackage{mathtools}
\usepackage{svg}

\date{\vspace{-4ex}}

\begin{document}
	\title{\textbf{Intrinsic Node Resistance and its Application in Depreciatory Adversarial Attacks}}
	\author{\emph{Ghanbar Azarnia, Sobhan Haghi}}
	\maketitle
	
	\large{Engineering Faculty of Khoy, Urmia University of Technology}\\
	
	\begin{abstract}
		Edge-weighted graphs are broadly utilized in the analysis of path-finding and optimization algorithms. Yet when it comes to network robustness, little attention is given to the properties of vertices. In this paper, we have introduced a novel network model that assigns an intrinsic resistance parameter to each node, these parameters are extracted from a uniform distribution on the set [0,1]. Introduction of such nodal parameters aids us in measuring overall network resilience. We also generalize an existing model governing network robustness measurements known as the Effective Node Resistance (abbreviated henceforth as ENR) to networks containing vertices with internal resistances. We will later provide state-of-art definitions for network attacker systems, specifically a generalization of these systems we call the Depreciatory Attacker Systems. Finally, the effectiveness of our proposed model in comparison to previous network models on defense against various types of depreciatory adversarial attacks will be analyzed through computer-assisted simulations.
	\end{abstract}
	
	\subparagraph{keywords}\textit{network security, network robustness, intrinsic node resistance, depreciatory adversarial attacks, attacker systems, graph theory}
	
	\section{Introduction}
	Ever since Leonhard Euler's innovative solution to the K\"{o}nigsberg Bridges problem, graph theorists sought to apply this newborn branch of mathematics to other natural sciences. Due to their flexible and high-level abstract nature that are also found in other topological objects, graphs proved to be effective in simulation of numerous phenomena in which a set of elements (labeled as vertices or nodes in graph theory nomenclature) are related by a predefined property\cite{tucker2012applied}. As a consequent of great surge in the number of internet users and the amelioration of computational power at the beginning of 21st century, graph theory has become even more popular tool to study complex networks such as the World Wide Web\cite{albert2000error}. A graph --here we may articulate the terms \textit{graph} and \textit{network}, \textit{vertex} and \textit{node} interchangeably--is symbolically represented as $G(V,E)$ where $V$ and $E$ are, respectively, the set of all vertices and the set of ordered pairs of some vertices called edges. Most models employ additional parameters known as weights to each edge of the graph forming an extended representation $G(V,E,w)$. These weights represent unique quantities for different fields but generally are related to physical properties such as distance, electrical resistance, spring stiffness etc. Weighted graphs are vastly utilized in the analysis of path-finding and optimization algorithms, yet when it comes to network robustness, little attention is given to the properties of vertices. Weights added to the vertices--simply called vertex weight--also exist but are not commonly employed as only a small set of problems (often related to computer science) require weighted vertices. Following is a small compilation of research papers involving graphs with weighted vertices.\newline \textit{Ramos et al.}\cite{ramos2021node} have assigned resistances to each node in order to represent a group of individuals' willingness to accept bribes. \textit{Chang et al.}\cite{chang2021rho} have defined $\rho$-moments for vertex-weighted graphs. \textit{\c{C}etinay et al.}\cite{cetinay2018nodal} have proposed assigning to the nodes (representing electrical buses in power grids) their respective centrality parameter in an attempt to model the vulnerability of power grids to targeted attacks while considering power flow dynamics of the system. \textit{Cavallaro et al.}\cite{ProbabNodeFailure} have developed a score-based system for their Probabilistic Node Failure model, such that a number of high-scoring nodes get deleted with respect to a threshold.\newline Other prominent disciplines that depend on graph theoretic simulations and illustrations are machine learning and artificial neural networks, physical chemistry\cite{klein1993m}, biology\cite{mason2007graph}\cite{knisley2014vertex}, fundamental mathematical concepts such as set theory, propositional logic and category theory, electrical engineering and circuitry, sociology, subway routes and traffic routing \cite{guze4805898continuous}, oil pipelines\cite{Pipeline} etc. In this paper, we will discuss network security from a graph theoretic point of view. Here, we shall introduce a novel network model that assigns an intrinsic resistance parameter ($\rho$) to each node which will aid us in measuring overall network resistance. We will analyze how introduction of the internal resistances, as an alternative to edge resistances, affects their resilience against adversarial attacks. Later, we will also provide formal definitions on attacker systems, specifically a generalization of these systems we call as the Depreciatory Attacker Systems and discuss these attacks are conducted on simple undirected networks.
	
	
	\section{Preliminaries}
	There are two major schools of robustness analysis i.e Classical Graph theory and Spectral Graph theory. The former group is mainly based on the topological structure of a network of $n$ vertices. For example, the Vertex Connectivity $K_v$ is defined as the minimal number of vertices to be removed such that the graph becomes disconnected. A similar Edge Connectivity $K_e$ also exists. It is said the higher $K_v$ (or $K_e$) higher the robustness gets \cite{ellens2013graph}. Another classical robustness measure is the Clustering coefficient C such that \cite{watts1998collective}:
	\begin{equation*}
		C = \frac{1}{n} \sum_{i\in V;\delta_i>1}\frac{2}{\delta_i(\delta_i -1)}e_i
	\end{equation*}
	where $e_i$ is the number of edges among neighbors of $i$ and $\delta_i$ degree of node $i$. Again it is trivial to consider as C gets larger, the network is more robust.\newline As the earliest attempt of node evaluation numerous centrality measures have been devised to distinguish vertices based on their importance in the robustness of their respective graphs. The most archaic centrality measure, namely \textbf{Degree Centrality}, functions solely by counting the degree of each vertex. Another promising measure is called the \textbf{Betweenness Centrality} and exists as an indicator of how popular a node is by counting the number of times the node appears in short-paths between other pairs of nodes.\cite{lee2021betweenness}\cite{chen2013node}
	\begin{equation*}
		B(v_i) = \sum_{s\neq t\neq v_i}\frac{\sigma_{st}(v_i)}{\sigma_{st}}
	\end{equation*}
	Where $\sigma_{st}$ indicates the number of all shortest paths between two arbitrary nodes $s$,$t$ and $\sigma_{st}(v_i)$ is the number of said shortest-paths that include $v_i$.\newline Two similar-looking measures are the \textbf{Closeness Centrality} $C_B(v)$ and its generalization to disconnected graphs the \textbf{Harmonic Centrality}\cite{marchiori2000harmony}\cite{rochat2009closeness} $H(v)$. Taking $d(u,v)$ as the distance between $u$ and $v$ while counting each edge as a distance unit of 1 we have:
	\begin{equation*}
		\begin{aligned}
			&C_B(v)= \frac{1}{\sum_{u}d(u,v)} \\
			&H(v) = \sum_{u|u\neq v} \frac{1}{d(u,v)}
		\end{aligned}
	\end{equation*}
	Additional methods such as Katz, Freeman, Eigenvector centrality, Cross-clique centrality exist but are not as prevalent as the ones mentioned above.
	Klein and Randi\'{c} \cite{klein1993m} have taken an electrical approach to analyze network robustness. For a pair of nodes such as (i,j), hypothetical 1 $\Omega$ resistances across (i,j) and a hypothetical battery parallel to (i,j) are wired. Thence an effective resistance between (i,j) can be obtained from equivalent resistance circuits.
	The spectral graph analysis utilizes various linear algebraic tools to quantify the characteristics of a particular graph. It seeks to relate the eigenvectors and eigenvalues of matrices corresponding to a Graph to the combinatorial properties of the graph. In the spectral theory, a graph is represented by an Adjacency matrix ($A$) whose $[A_{ij}]$th entry equals to 1 one if an edge connects the nodes $i$ and $j$ denoted as $i\sim j$.
	\begin{equation*}
		A_{i,j} = \begin{cases}
			1	& i\sim j \\
			0	& o.w. 
		\end{cases}
	\end{equation*}
	Accompanying the Adjacency matrix, a Degree matrix is also introduced that contains only diagonal entries bearing the number of neighboring vertices $\delta(v_i)$ for $i$th vertex.
	\begin{equation*}
		D_{i,j} = \begin{cases}
			\delta(v_i) & i=j \\
			0 & o.w.
		\end{cases}
	\end{equation*}
	Another highly functional matrix derived from these two matrices is the Laplacian ($L$) matrix constructed by: $L=D-A$. Many ideas in spectral analysis have come from facts already known in electrical circuit theory. Three of the most common methods are the number of Spanning trees, algebraic connectivity (the second smallest eigenvalue of the Laplacian matrix) and the Effective Graph Resistance \cite{ellens2013graph}. The latter method is of special importance to our research for it lays the foundations of later ideas we have used in our model. 
	Assuming each edge of the network corresponds to a 1$\Omega$ resistor, the Effective Graph Resistance (also known as the Kirchhoff Index), is defined as the sum of equivalent resistances over all pairs of nodes. It is expressed as
	$R = n \sum_{i=2}^{n}(\frac{1}{\lambda_{i}})$, where $n = |V|$ is the number of vertices in the network\cite{klein1993m}\cite{yamashita2021effective}. Note that $\bigcup_{i=1}^{n} \{\lambda_i\}$ forms the set of all real eigenvalues of the Laplacian matrix. The lower bound of summation must indeed begin from $i=2$ since $\lambda_1 = 0$.
	
	\section{Effective Node Resistance (ENR)}
	For any pair of vertices such as (i,j), W.Ellens \cite{ellens2013graph} has shown that an effective resistance can be defined as:
	\begin{equation}
		R_{ij} = (e_i-e_j)^T  L^{-1} (e_i-e_j) = L_{ii}^{\dag} -2L_{ij}^{\dag} + L_{jj}^{\dag}= R_{ji}
	\end{equation}
	Where $e_i$ is the basis vector that has zeros everywhere except a one at its $i$th entry. Since the Laplacian has one 0 eigenvalue at $\lambda_1$, taking its inverse is a mathematical impossibility. This error can be circumvented by replacing $L^{-1}$ with the Moore-Penrose inverse operation $L^\dag$. From the relation above an equation for the total effective node resistance (ENR) of $i_{th}$ vertex is derived as $R_i = \sum_{j\in V}R_{ij} = \sum_{j\in V/ \{i\}} R_{ij}$. This is, however, different than the total effective resistance $R_{tot}= \sum_{i=1}^{n}\sum_{j=i+1}^{n}R_{ij} = \frac{1}{2}\sum_{i=1}^{n}\sum_{j=1}^{n}R_{ij}$ as the latter is given by the sum of the effective resistances over
	all pairs of vertices and is an attribute of the whole network whereas the ENR is uniquely calculated for each node.
	
	\subsection{ENR given random Intrinsic Resistances}
	Suppose we map each node in the network to a random variable $\rho$ as its Intrinsic Resistance where $\rho$ is drawn from uniform distribution on [0,1], $\rho \sim U[0,1]$. A formal definition would be to consider $\rho$ as a function that maps each vertex to a number in [0,1], $\rho: V \rightarrow \left[ 0,1 \right]$. Hence a new formula for the ENR can be derived as:
	\begin{equation}
	\begin{aligned}
		&R_{ij} = R_{ji} = (\rho_i e_i - \rho_j e_j)^T L^{\dag} (\rho_i e_i - \rho_j e_j), \\
		&=\rho_i e_i^{T}L^{\dag}\rho_i e_i -\rho_j e_j^{T}L^{\dag}\rho_i e_i - \rho_i e_i^{T}L^{\dag}\rho_j e_j + \rho_j e_j^{T}L^{\dag}\rho_j e_j , \\
		&= \rho_i^{2} L_{ii}^{\dag} - 2\rho_i \rho_j L_{ij}^{\dag} + \rho_j^{2} L_{jj}^{\dag}
	\end{aligned}
	\end{equation}
	Note that the equation (1) is merely a special case of (2) in which $\rho_i = \rho_j = 1, \forall i,j \in V$. The reason behind introduction of Internal Resistances is to accurately simulate real-world relationships in which the quantity or quality of agent-to-agent connections per se is of little to no importance. As an example, suppose we have a network of interconnected computers scattered all around the globe. It is foolish to consider that the distances between these computers contribute anything to their immunity against malicious attacks. The decisive factor over the security of this network is the tolerance of each computer towards various forms of adversarial attacks. Another application for this method can be found in criminology and riot-control where to apprehend the leaders of a criminal organization one needs to first identify, apprehend, and interrogate fragile members of the organization. We believe our model would potentially complement the existing epidemiological models such as the SIR (susceptible-infectious-removed) model, since every susceptible individual in the simulated world can be considered as nodes with unique resistances to infection. Further explanation and examples can be given whence we have completed our discussion on attacker systems. Based on the proposed model (2), we derive a new equation for the total effective resistance of the network
	\begin{equation}
		\begin{aligned}
			&R_{tot}^{\star}=\frac{1}{2}\sum_{i=1}^{n}\sum_{j=1}^{n} \left(\rho_i^{2} L_{ii}^{\dag} - 2\rho_i \rho_j L_{ij}^{\dag} + \rho_j^{2} L_{jj}^{\dag} \right) \\
			&= n\sum_{i=1}^{n} \rho_i^{2}L_{ii}^{\dag} - \sum_{i=1}^{n}\sum_{j=1}^{n} \rho_i\rho_j L_{ij}^{\dag}
		\end{aligned}
	\end{equation} 
	Also by inserting (2) at the total ENR formula we get:
	\begin{equation}
	\begin{aligned}
		&R_i = \sum_{j\in V/ \{i\}} \left( \rho_i^{2} L_{ii}^{\dag} - 2\rho_i \rho_j L_{ij}^{\dag} + \rho_j^{2} L_{jj}^{\dag} \right) \\
		&=(n-1)\rho_i^{2}L_{ii}^{\dag}-2\rho_i \sum_{j\in V/ \{i\}} \rho_j L_{ij}^{\dag} + \sum_{j\in V/ \{i\}} \rho_j^{2} L_{jj}^{\dag}
	\end{aligned}
	\end{equation}
	We may now discuss whether higher internal resistance implies higher ENR. If perceived as a function of $\rho_i$, $R_i$ is strictly ascendant if and only if $\forall \rho_i, \frac{dR_i}{d\rho_i} > 0$:
	\begin{equation}
		\frac{dR_i}{d \rho_i} = 2(n-1)\rho_i L_{ii}^{\dag} - 2\sum_{j\in V / \{i\}} \rho_j L_{ij}^{\dag} > 0
	\end{equation}
	Here we face a dilemma, it is well-known that the diagonal entries of Laplacian matrix ($L_{ii}^{\dag}$) are always positive, thus the first term on the LHS of (5) will cause no trouble regarding its sign. However, as shown in Fig 1, for some graphs, the off-diagonal entries ($L_{ij}^{\dag}$) are all negative and for some there exists at least one off-diagonal positive entry. Hence no concrete deduction can be made whether (5) always holds. By altering the above relation, we deduced the following criterion that $\rho_i$ must abide by:
	\begin{equation}
		\frac{\sum_{j\in V / \{i\}} \rho_j L_{ij}^{\dag}} {(n-1)\rho_i L_{ii}^{\dag}} < 1 \Longleftrightarrow \rho_i > \frac{\sum_{j\in V / \{i\}} \rho_j L_{ij}^{\dag}} {(n-1) L_{ii}^{\dag}}
	\end{equation}
	\newline
	\begin{figure}[t!]
		\centering
		\begin{subfigure}[t]{0.4\textwidth}
			\centering
			\includesvg[scale=0.65]{G1.svg}
			\caption{$G_1$: A graph with all negative off-diagonal elements on its $L^{\dag}$}
			\label{fig:3-a}
		\end{subfigure}
		\hfill
		\begin{subfigure}[t]{0.4\textwidth}
			\centering
			\includesvg[scale=0.5]{G2.svg}
			\caption{$G_2$: A graph with some positive off-diagonal elements on its $L^{\dag}$}
			\label{fig:3-b}
		\end{subfigure}
		\caption{Two arbitrary graphs}
		\label{fig:3}
	\end{figure}
	
	\hspace{2pt}
	
	\[
	L^{\dag}_{G_1}=
	\begin{bmatrix}
		0.160 & -0.040 & -0.040 & -0.040 & -0.040 \cr
		-0.040 &  0.236 & -0.002 & -0.050 & -0.145 \cr
		-0.040 & -0.002 &  0.379 & -0.145 & -0.192 \cr
		-0.040 & -0.050 & -0.145 &  0.236 & -0.002 \cr
		-0.040 & -0.145 & -0.192 & -0.002 &  0.379
	\end{bmatrix}
	\]
	
	\setlength{\leftskip}{-3cm}
	\begin{equation*}
		\hspace*{-3cm}
		L^{\dag}_{G_2}=
		\begin{bmatrix}
			0.345 &  0.039 & -0.087 & -0.087 &  0.039 &  0.055 & -0.039 & -0.113 & -0.113 & -0.039 \cr
			0.039 &  0.345 &  0.039 & -0.087 & -0.087 & -0.039 &  0.055 & -0.039 & -0.113 & -0.113 \cr
			-0.087 &  0.039 &  0.345 &  0.039 & -0.087 & -0.113 & -0.039 &  0.055 & -0.039 & -0.113 \cr
			-0.087 & -0.087 &  0.039 &  0.345 &  0.039 & -0.113 & -0.113 & -0.039 &  0.055 & -0.039 \cr
			0.039 & -0.087 & -0.087 &  0.039 &  0.345 & -0.039 & -0.113 & -0.113 & -0.039 &  0.055 \cr
			0.055 & -0.039 & -0.113 & -0.113 & -0.039 &  0.345 &  0.039 & -0.087 & -0.087 &  0.039 \cr
			-0.039 &  0.055 & -0.039 & -0.113 & -0.113 &  0.039 &  0.345 &  0.039 & -0.087 & -0.087 \cr
			-0.113 & -0.039 &  0.055 & -0.039 & -0.113 & -0.087 &  0.039 &  0.345 &  0.039 & -0.087 \cr
			-0.113 & -0.113 & -0.039 &  0.055 & -0.039 & -0.087 & -0.087 &  0.039 &  0.345 &  0.039 \cr
			-0.039 & -0.113 & -0.113 & -0.039 &  0.055 &  0.039 & -0.087 & -0.087 &  0.039 &  0.345
		\end{bmatrix}
	\end{equation*}
	\setlength{\leftskip}{0pt}
	
	\section*{Corollary}
	In extremely rare scenarios where most of off-diagonal entries in the inverse Laplacian are positive, the equation (3) for $R^{\ast}_{tot}$ may become negative if and only if its second term is greater than its first term i.e. $\sum_{i=1}^{n}\sum_{j=1}^{n} \rho_i\rho_j L_{ij}^{\dag} > n\sum_{i=1}^{n} \rho_i^{2}L_{ii}^{\dag}$. We believe this inequality rarely holds since the multiple $n$ on the RHS will most of the time yield a sufficiently large number. Here we must search for a graph Laplacian whose inverse contains mostly positive entries. As of today, no efficient method exists to gain insight on the distribution of entries in $A^{-1}$ given $A$. Without any difficulties one may declare that this problem is in NP. Designing an algorithm to solve the problem is beyond the scope of this paper, instead we propose usage of iterative methods \cite{corderoinverse}\cite{ZAKAULLAH2014441} may prove valuable.
	
	\subsection{Defense scheme based on Internal Resistances}
	Here we build a defense scheme against adversarial attacks based on these assumptions:
	
	\begin{enumerate}
		\item {We only deal with centralized, simple, connected, and undirected networks.}
		Centrality-wise, we will take an unorthodox approach. We say a network is centralized if it has at least one node whose internal resistance is approximately 1. These are to be labeled as Leaders (denoted $\rho_L$) whose elimination would results in total failure of the network. If $l$ is the set of all leader nodes then for our model the maximum possible cardinality of this set is $|l| = n-1$, since $|l|=n$ gives back the original equation (1)
		\item All networks discussed only have one leader
		\item {The network fails to defend itself if the leader gets removed (undesirable state)}
		\item {Our results apply to both scale-free and random networks}
	\end{enumerate}
	
	\subsection{Depreciatory Attacker Systems}
	An attacker system is simply any algorithm or protocol used to eliminate a set of nodes at a given time. The most classical node elimination method is the blind random removal where the number of nodes to be eliminated is not bounded by a deterministic or semi-deterministic procedure. The application of no restrictions on the operation of such systems may prove them incompetent for real-world applications where internal node resistances play an essential role in the network robustness. Thus fundamental changes are needed in our definition of an attacker systems. A Depreciatory Attacker System depends on two parameters: $G^{(t)}$, the perturbed network and $F_{t}$, the remaining fuel. From this moment onward we will formally denote our DAS model using the Hebrew letter ($\aleph$) as a 2-tuple consisting of these two parameters:
	$\aleph =(G^{(t)},F_t)$. We choose the initial $F_0$ as an arbitrarily large number ($ \geq 100$) and thence compute the cost function $K$ that models the gradual decrease in Fuel for each epoch. As an example, let $K(\langle \rho \rangle,F_{t})$ be a linear function such that $K(\langle \rho \rangle,F_{t})= F_{t+1} = F_{t} - \langle \rho \rangle F_{t}$ we will use $\langle \rho \rangle = \sum_{\rho \in \{ \rho \}}\rho/| \{ \rho \}|$ to denote average of the set containing internal resistances of the remaining nodes.
	\textit{Cavallaro et al.}\cite{ProbabNodeFailure} have proposed a Node Survival Probability function $\psi:V\rightarrow [0,1]$ that returns the probability $\psi(i)$ that node $i\in V$ survives a failure. Their proposed model for this function follows from the intuitive statement that nodes with larger degrees are susceptible to removal and disrupt network integrity. Hence they have picked the Survival Probability function $\psi(i)$ to be a measure of how densely connected $i$ is over all nodes present in the network.
	\begin{equation*}
		\psi(i) = \frac{\delta(i)}{|V|}
	\end{equation*}
	The score function $\sigma(i)$ is associated to each node such that $\sigma(i) = [1-\psi(i)]\phi(i)$ where $\phi(i)$ is the measured centrality of $i$. The next step is to remove $\lceil \tau|N_0| \rceil$ \footnote{$\tau \in [0,1]$ is called the target threshold.} nodes with largest scores from the network. Therefore high-scoring nodes are the ones with larger centrality and larger probability of failure. Inspired by their research, we propose the following node elimination process. Before we proceed to discuss our experiments, the following theorem must be introduced to better understand the reason behind the results.
	\newtheorem{thm}{Theorem}
	\begin{thm}\label{disttheorem}
		For two finite sets with equal cardinalities such as $X=\{x|x\sim U[a,b]\}$ and $Y=\{y|y\sim U[c,d]\}$, if $a<c$ and $b<d$ then $\langle X\rangle < \langle Y\rangle $
	\end{thm}
	We do not attempt to provide a proof for this theorem as it is self-explanatory
	\section{Experiment and Simulation}
	Our method involves removing $\lfloor \tau(F_t)\rfloor$ nodes at each epoch. $\tau(F_t)$ is a function of remaining fuel whose floored output dictates the number of nodes to be removed and is expected to be decreasing as $F_t \rightarrow 0$. Thus, for our experimentation, we have concluded that $\tau(F_t)=a_{t}(1-e^{-\kappa F_t})+b$ fits our expectations given $a_t = \langle \rho \rangle$, $b = \rho_L =1$ and $\kappa=0.05$.
	To show how perturbed the resulting graph has become (i.e the final form after removal is done) we will utilize two measures introduced in \cite{ProbabNodeFailure} namely the Coverage: $\gamma = \frac{\widetilde C}{C}$ and Effectiveness: $\epsilon = \frac{\widetilde \lambda_1}{\lambda_1}$ whereas $C$ and $\lambda_1$ denote the size of Largest Connected Component and the Spectral Radius (the largest eigenvalue of the Adjacency matrix), the tilde on top represents these measures computed on the perturbed graph. We have chosen the initial Fuel $F_0=100$ and node-to-node connection probability $p=\frac{1}{3}$ as constants for all our experiments. The number of epochs passed until successful removal of the leader is also considered throughout the experiment and will be shown by $\textit{\textbf{t}}$ on some charts or tables.
	\subsection{methodology}
	  For each random graph generation model (Erd\H{o}s-Renyi and Albert-Barb\'{a}si) four sets of experiments have been conducted with four different Uniform Distributions from which the internal resistances are yielded: $U(0,0.33)$ \& $U(0.33,0.66)$ \& $U(0.66,1)$ \& $U(0,1)$. The latter is serves a benchmark. \newline Then, for each distribution, aforementioned measures are computed and averaged for fifty consecutive runs. A considerable number of these fifty repetitions will result in failure when the DAS runs out of fuel before the leader is removed, we have also reported the Averaged Attack Failure Percentage (Fig.2) by indexing the number of failed attempts five times and computing their average. \newline
	  \begin{figure}[tbh]\label{fig:2}
	  	\centering
	  	\includesvg[scale=0.75]{Average-Epoch.svg}
	  	\caption{$<\textbf{t}>$ vs. $|N|$}
	  \end{figure} \newline
	  Our simulation results report a rather unexpected outcome. For networks with smallest distribution of Internal Resistances ($\rho \sim U[0,0.33]$) it takes much longer epochs to eliminate the leader in comparison to other three distributions. Note also that the average number of epochs had an ascending rate of change until $|N|=75$ before it dropped out. To explain this bizarre behavior we have also plotted the Average Failure Percentage (Fig.3) with respect to $|N|$ 
	  \begin{figure}
	  	\centering
	  	\includesvg[scale=0.75]{Average-Attack-Failure-Percentage.svg}
	  	\caption{Network robustness is proportional to $|N|$}
	  \end{figure} \newline
	  Here we demonstrate that the network success rate depends on both $|N|$ and $<\rho>$. From \textbf{Theorem} \ref{disttheorem} we know if $\rho_1 = \{\rho \sim U[0,0.33] \} < \rho_2 = \{\rho \sim U[0.33,0.66] \} < \rho_3 = \{\rho \sim U[0.66,1] \}$ then we have
	  \begin{equation*}
	  	\langle \rho_1 \rangle < \langle \rho_2 \rangle < \langle \rho_3 \rangle
	  \end{equation*}
	   Henceforth it is inferred that networks whose Internal Resistances were assigned from $\rho_3$ had greater decrease in fuel per our definition of $K(\langle \rho \rangle,F_{t})$. This property aids us at explaining (Fig 2): A network with Internal Resistances assigned from $\rho_1$ will posses a cost function of shallowest slope (i.e. $\langle \rho_1 \rangle$) and will continuously get shallower hence saving more fuel for next epochs. However this network performs poorly at defense (Fig 3)
	\pagebreak
	\bibliographystyle{elsarticle-num}
	\bibliography{NodeRef}
\end{document}
